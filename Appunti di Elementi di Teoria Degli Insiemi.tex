\documentclass[11pt]{scrartcl}
\usepackage[italian]{babel}
\usepackage[sexy]{evan} %evan.sty

%%%%%%%%%%%%%%%%%%%%%%%%%%%%%%%%%%%%%%%%%%%%%%%%%%%%%%%%%%%%%%%%%%%%%%%%%%%%%%
%
% BOOST SOFTWARE LICENSE - VERSION 1.0 - 17 AUGUST 2003
%
% Copyright (c) 2022 Evan Chen [evan at evanchen.cc]
% https://web.evanchen.cc/ || github.com/vEnhance
%
% Available for download at:
% https://github.com/vEnhance/dotfiles/blob/main/texmf/tex/latex/evan/evan.sty
%
% Permission is hereby granted, free of charge, to any person or organization
% obtaining a copy of the software and accompanying documentation covered by
% this license (the "Software") to use, reproduce, display, distribute,
% execute, and transmit the Software, and to prepare derivative works of the
% Software, and to permit third-parties to whom the Software is furnished to
% do so, all subject to the following:
%
% The copyright notices in the Software and this entire statement, including
% the above license grant, this restriction and the following disclaimer,
% must be included in all copies of the Software, in whole or in part, and
% all derivative works of the Software, unless such copies or derivative
% works are solely in the form of machine-executable object code generated by
% a source language processor.
%
% THE SOFTWARE IS PROVIDED "AS IS", WITHOUT WARRANTY OF ANY KIND, EXPRESS OR
% IMPLIED, INCLUDING BUT NOT LIMITED TO THE WARRANTIES OF MERCHANTABILITY,
% FITNESS FOR A PARTICULAR PURPOSE, TITLE AND NON-INFRINGEMENT. IN NO EVENT
% SHALL THE COPYRIGHT HOLDERS OR ANYONE DISTRIBUTING THE SOFTWARE BE LIABLE
% FOR ANY DAMAGES OR OTHER LIABILITY, WHETHER IN CONTRACT, TORT OR OTHERWISE,
% ARISING FROM, OUT OF OR IN CONNECTION WITH THE SOFTWARE OR THE USE OR OTHER
% DEALINGS IN THE SOFTWARE.
%%%%%%%%%%%%%%%%%%%%%%%%%%%%%%%%%%%%%%%%%%%%%%%%%%%%%%%%%%%%%%%%%%%%%%%%%%%%%%

\begin{document}
\title{Elementi Di Teoria Degli Insiemi}
\subtitle{\large\normalfont\rmfamily\scshape APPUNTI DEL CORSO DI ELEMENTI DI TEORIA DEGLI INSIEMI \\ TENUTO DAL PROF. MARCELLO MAMINO}
\author{Diego Monaco \\ \textnormal{\href{d.monaco2@studenti.unipi.it}{d.monaco2@studenti.unipi.it}} \\ Università di Pisa}
\date{Anno Accademico 2022-23}
\maketitle
\newpage

\tableofcontents
\eject
\newpage

\section*{Premessa}
\section*{Ringraziamenti}

\mbox{}
\vfill
\begin{wrapfigure}{R}{0.2\textwidth}
	\centering
	\href{https://creativecommons.org/licenses/by-nc/4.0/deed.it}{\includegraphics[width=0.2\textwidth]{licenza.png}}
\end{wrapfigure}

Quest'opera è stata rilasciata con licenza Creative Commons Attribuzione - Condividi allo stesso modo 4.0 Internazionale. Per leggere
una copia della licenza visita il sito web \href{http://creativecommons.org/licenses/by-sa/4.0/deed.it}{\textcolor{blue}{https://creativecommons.org/licenses/by-nc/4.0/deed.it}}.\\

\newpage

\newpage
\section{Prologo nel XIX secolo}
La nascita della teoria degli insiemi è una storia complicata di cui so pochissimo. Però, persone che ne sanno molto più di me hanno sostenuto l'opinione che il problema seguente
abbia avuto un ruolo. Come che sia, è almeno un'introduzione possibile.

\begin{problem}
Data una serie trigonometrica:
\[ S(x) = c_0 + \sum_{i=1}^{+\infty}a_i\sin{(ix)}+b_i\cos{(ix)}
	\]
se, per ogni $x \in \RR$, sappiamo che $S(x)$ converge a 0, possiamo dire che i coefficienti $c_0,a_i,b_i$ sono tutti 0?
\end{problem}

Risolto positivamente da \href{https://it.wikipedia.org/wiki/Georg_Cantor}{\textcolor{purple}{Georg Cantor}} nel 1870.

\begin{definition}
Diciamo che $X \subseteq \RR$ è un \vocab{insieme di unicità} se, per ogni serie trigonometrica:
\[ S(x) = c_0 + \sum_{i=1}^{+\infty}a_i\sin{(ix)}+b_i\cos{(ix)}
	\]
vale la seguente implicazione:
\[ \text{$S(x)$ converge a 0 per tutti gli $x\not\in X$} \implies \text{tutti i coefficienti $c_0,a_i,b_i$ sono nulli}
	\]
\end{definition}

\begin{example}
	Per il risultato di Cantor, $\emptyset$ è di unicità.
\end{example}

\begin{problem}
	Quali sottoinsiemi di $\RR$ sono di unicità?
\end{problem}

\begin{fact}
\label{unicità}
$X \subseteq \RR$ è di unicità se (ma non solo se) ogni funzione continua $f : \RR \longrightarrow \RR$ che soddisfi le ipotesi seguenti è necessariamente lineare\footnote{$f(x) = \alpha x + \beta$.}:
\begin{itemize}
	\item per ogni intervallo aperto $\left]a,b\right[$ con $]a,b[ \cap X = \emptyset$, $f_{|\left]a,b\right[}$ è lineare;
	\item per ogni $x \in \RR$, se $f$ ha derivate destre e sinistre in $x$, allora queste coincidono\footnote{Ovvero $f$ non ha punti angolosi.}.
\end{itemize}
\end{fact}

\begin{example}
	$X = \{\ldots,a_{-2},a_{-1},a_0,a_1,a_2,\ldots\} = \{a_i | i \in \ZZ\}$ con $\ldots < a_{-2} < a_{-1} < a_0 < a_1 < a_2 <\ldots$, $\displaystyle\lim_{i \to +\infty} a_i = +\infty$, $\displaystyle\lim_{i \to -\infty} a_i = -\infty$ ha la 
	proprietà data dal \hyperref[unicità]{Fatto 1.5}, quindi è di unicità.
\end{example}

\begin{notexample}
L'intervallo $[0,1]$ o $\RR$ non hanno la proprietà espressa dall'\hyperref[unicità]{Fatto 1.5}.
\end{notexample}

\begin{notexampleb}
Per l'\vocab{insieme di Cantor} non vale il \hyperref[unicità]{Fatto 1.5}.
\end{notexampleb}

Possiamo costruire l'insieme di Cantor a partire dall'intervallo $C_0 = [0,1]$ nel seguente modo:

\begin{center}
	\begin{figure}[h]
		\centering
		\includegraphics[width=12.5cm]{immagini/cantor.png}
	\end{figure}
\end{center}

ovvero, preso l'intervallo $[0,1]$ possiamo dividerlo in tre parti e rimuovere la parte centrale $\displaystyle\left(\frac 13, \frac 23\right)$, chiamiamo gli intervalli rimanenti $C_1$, possiamo iterare il procedimento sui due segmenti di $C_1$ ed ottenere $C_2,C_3,\ldots$, a questo punto 
definiamo l'insieme di Cantor $C$ come:
\[ C := \bigcap_{i \in \NN}C_i
	\]
Esiste una funzione continua $f : \RR \longrightarrow \RR$ detta \vocab{scala di Cantor} (o \vocab{scala del diavolo}), tale che $f^{\prime}(x) = 0$ per $x \not\in C$ e non è 
derivabile in $x \in C$.

\subsection{Digressione: insiemi numerabili}
\begin{definition}
	Un insieme $X$ è \vocab{numerabile} se è il supporto di una successione, $X = \{a_0,a_1,a_2,\ldots\} = \{a_i | i \in \NN\}$, con $a_i \ne a_j$ per ogni $i \ne j$.
\end{definition}

\newpage
\subsection{Tornando agli insiemi di unicità}
\newpage
\subsection{Giochi di parole}
\newpage
\subsection{Scopi del corso}


\newpage
\section{Il linguaggio della teoria degli insiemi}
\subsection{Le regole di inferenza}



\newpage
\section{I primi assiomi}
\begin{axiom}
[Assioma dell'insieme vuoto]
Esiste un insieme vuoto.
\[ \exists x, \forall y : y \not\in x
		\]
\end{axiom}

\begin{axiom}
[Assioma di estensionalità]
Un insieme è determinato dalla collezione dei suoi elementi. Due insiemi coincidono se e solo se hanno i medesimi elementi.
\[ \forall a, \forall b : a = b \longleftrightarrow \forall x (x \in a \longleftrightarrow x \in b)
	\]
\end{axiom}

\subsection{Assiomi dell'insieme vuoto e di estensionalità}
\newpage
\subsection{Assioma di separazione}
\newpage
\subsection{Classi e classi proprie}
\newpage
\subsection{Assioma del paio e coppia di Kuratowski}
\newpage
\subsection{Assioma dell'unione e operazioni booleane}
\newpage
\subsection{Assioma delle parti e prodotto cartesiano}
\newpage
\subsection{Relazioni di equivalenza e di ordine, funzioni}



\newpage
\section{Assioma dell'infinito e numeri naturali}
\subsection{Gli assiomi di Peano}
\newpage
\subsection{L'ordine di omega}
\newpage
\subsection{Induzione forte e principio del minimo}
\newpage
\subsection{Ricorsione numerabile}




\newpage
\section{Cardinalità}
\subsection{Teorema di Cantor-Berstein}
\newpage
\subsection{Teorema di Cantor}
\newpage
\subsection{Operazioni fra cardinalità}





\newpage
\section{Cardinalità finite}
\subsection{Principio dei cassetti}
\newpage
\subsection{Operazioni fra le cardinalità finite}




\newpage
\section{La cardinalità numerabile}
\subsection{Insiemi numerabili in pratica}
\newpage
\subsection{Prodotto di numerabili è numerabile}
\newpage
\subsection{Numeri interi e razionali}
\newpage
\subsection{Ordini densi numerabili}
\newpage
\subsection{Il grafo random}




\newpage
\section{I numeri reali e la cardinalità del continuo}
\subsection{Caratterizzazione dei reali come ordine}
\newpage
\subsection{La cardinalità del continuo è 2 alla alef-zero}
\newpage
\subsection{Operazioni che coinvolgono la cardinalità del continuo}
\newpage
\subsection{Sottrarre un numerabile dal continuo}




\newpage
\section{Stato del corso}




\newpage
\section{I buoni ordinamenti}
\subsection{Operazioni fra buoni ordinamenti}
\newpage
\subsection{Gli ordinali di Von Neumann}
\newpage
\subsection{Assioma del rimpiazzamento}
\newpage
\subsection{Induzione e ricorsione transfinita}
\newpage
\subsection{Operazioni fra gli ordinali}




\newpage
\section{Aritmetica ordinale e forma normale di Cantor}
\subsection{Sottrazione e divisione euclidea}
\newpage
\subsection{La forma normale di Cantor}
\newpage
\subsection{Punti fissi e epsilon-numbers}
\newpage
\subsection{Operazioni in forma normale di Cantor}
\newpage





\newpage
\section{Gli alef}
\subsection{Teorema di Hartogs}
\newpage
\subsection{Somme e prodotti di alef}



\newpage
\section{L'assioma della scelta}
\subsection{Buon ordinamento implica AC}
\newpage
\subsection{AC implica buon ordinamento (idea)}
\newpage
\subsection{Zorn implica buon ordinamento}
\newpage
\subsection{AC implica Zorn}
\newpage
\subsection{Conseguenze immediate di AC}
\newpage
\subsection{Esempi di applicazione di AC}
\newpage
\subsection{Basi di spazi vettoriali}
\newpage
\subsection{Invariante di Dehn}
\newpage
\subsection{Insieme di Vitali}
\newpage
\subsection{Teorema di Cantor-Bendixson}
\newpage
\subsection{Teorema di Tarski sulla scelta}









\newpage
\section{Aritmetica cardinale}
\subsection{Somme e prodotti infiniti}
\newpage
\subsection{Teorema di König}
\newpage
\subsection{Cofinalità}
\newpage
\subsection{Formula di Hausdorff}






\newpage
\section{Gerarchia di Von Neumann}
\subsection{Formule relativizzate ad una classe}
\newpage
\subsection{Assioma di buona fondazione}
\newpage
\subsection{Principio di epsilon-induzione}


\end{document}